\chapter{Partial Differential Equations}
\section{Introduction}
Partial Differential Equations (\addtoindex{PDE}), occur frequently in maths, natural science
and engineering.\\
\addtoindex{PDE}'s are problems involving rates of change of functions of several variables.\\
The following involve 2 independent variables:
\begin{eqnarray*}
-\nabla^2u=-\frac{\partial^2u}{\partial x^2}-\frac{\partial^2u}{\partial y^2}=f(x,y)&\mbox{Poisson Eqn}\\
\frac{\partial u}{\partial t}+v\frac{\partial u}{\partial x}=0&\mbox{Advection Eqn}\\
\frac{\partial u}{\partial t}-D\frac{\partial^2 u}{\partial x^2}=0&\mbox{Heat Eqn}\\
\frac{\partial^2 u}{\partial t^2}-c^2\frac{\partial^2 u}{\partial x^2}=0&\mbox{Wave Equation}
\end{eqnarray*}
Here $v,D,c$ are real positive constants.  In these cases $x,y$ are the space
coordinates and $t,x$ are often viewed as time and space coordinates, respectively.\\
These are only examples and do not cover all cases.  In real occurrences \addtoindex{PDE}'s
usually have 3 or 4 variables.\\
\section{PDE Classification}
\addtoindex{PDE}'s in two independent variables x and y have the form
\[ \Phi\left(x,y,u,\frac{\partial u}{\partial x},\frac{\partial u}{\partial y},\frac{\partial^2 u}{\partial x^2},...\right) = 0 \]
where the symbol $\Phi$ stands for some functional relationship.\\
As we saw with BVP this is too general a case so we must define new classes of the
general \addtoindex{PDE}.
\begin{definition}
The order of a \addtoindex{PDE} is the order of the highest derivative that appears.\\
ie Poisson is 2nd order, Advection eqn is 1st order. $\circ$
\end{definition}
Most of the mathematical theory of \addtoindex{PDE}'s concerns linear equations of first or second order.\\
After order and linearity (linear or non-linear), the most important classification
scheme for \addtoindex{PDE}'s involves geometry.\\
Introducing the ideas with an example:\\
\begin{example}
\begin{equation}\label{ex}
\alpha(t,x)\frac{\partial u}{\partial t}+\beta\frac{\partial u}{\partial x}=\gamma(t,x)\end{equation}
A solution $u(t,x)$ to this \addtoindex{PDE} defines a surface $\{t,x,u(t,x)\}$ lying over some region of the $(t,x)$-plane.\\
Consider any smooth path in the $(t,x)$-plane lying below the solution $\{t,x,u(t,x)\}$.\\  Such a path has a parameterization $(t(s),x(s))$ where the parameter
s measures progress along the path.\\
What is the rate of change $\frac{d u}{d s}$ of the solution as we travel along the path $(t(s),x(s))$.\\
The chain rule provides the answer
\begin{equation}\label{chain}\frac{d t}{d s}\frac{\partial u}{\partial t}+
\frac{d x}{d s}\frac{\partial u}{\partial x}
=\frac{d u}{d s}\end{equation}
Equation (\ref{chain}) holds for an arbitrary smooth path in the $(t,x)$-plane.  
Restricting attention to a specific family of paths leads to a useful observation:
When
\begin{equation}\frac{dt}{ds}=\alpha(t,x) \mbox{  and  } \frac{dx}{ds}=\beta(t,x)
\end{equation}
the simultaneous validity of (\ref{ex}) and (\ref{chain}) requires that
\begin{equation}\label{gamma}
\frac{du}{ds}=\gamma(t,x).
\end{equation}
Equation (\ref{gamma}) defines a family of curves $(t(s),x(s))$ called characteristic curves in the plane $(t,x)$.\\
Equation (\ref{gamma}) is an ode called the characteristic equation that the solution must satisfy along only the characteristic curve.\\
Thus the original \addtoindex{PDE} collapses to an ODE along the characteristic curves. Characteristic 
curves are paths along which information about the solution to the \addtoindex{PDE} propagates
from points where the initial value or boundary values are known.
$\diamond$\end{example}
Consider a second order \addtoindex{PDE} having the form
\begin{equation}
\alpha(x,y)\frac{\partial^2u}{\partial x^2}+\beta(x,y)\frac{\partial^2 u}{\partial x\partial y}+\gamma(x,y)\frac{\partial^2u}{\partial y^2} = \Psi(x,y,u,\frac{\partial u}{\partial x},\frac{\partial u }{\partial y})
\end{equation}
Along an arbitrary smooth curve $(x(s),y(s))$ in the $(x,y)$-plane, the gradient
$\left( \frac{\partial u}{\partial x}, \frac{\partial u}{\partial y}\right)$ of the solution varies according to the chain rule:
\[ \frac{dx}{ds}\frac{\partial^2u}{\partial y\partial x}+\frac{dy}{ds}\frac{\partial^2 u}{\partial y\partial x}=\frac{d}{ds}\left(\frac{\partial u}{\partial x}\right)\]
\[ \frac{dx}{ds}\frac{\partial^2u}{\partial x \partial y}+\frac{dy}{ds}\frac{\partial^2u}{\partial y^2}=\frac{d}{ds}\left(\frac{\partial u}{\partial y}\right)\]
if the solution $u(x,y)$ is continuously differentiable then these relationships
together with the original \addtoindex{PDE} yield the following system:
\begin{equation}
\label{matrix char} 
\left(\begin{array}{ccc}
\alpha & \beta & \gamma \\
\frac{dx}{ds}&\frac{dy}{ds}&0\\
0&\frac{dx}{ds}&\frac{dy}{ds}\\
\end{array}\right)
\left(\begin{array}{c}
\frac{\partial^2u}{\partial x^2}\\
\frac{\partial^2u}{\partial x\partial y}\\
\frac{\partial^2u}{\partial y^2}\\
\end{array}\right)
=
\left(\begin{array}{c}
\Psi\\
\frac{d}{ds}\left(\frac{\partial u}{\partial x}\right)\\
\frac{d}{ds}\left(\frac{\partial u}{\partial y}\right)\\
\end{array}\right)
\end{equation} 
By analogy with the first order case we determine the characteristic curves bu where the \addtoindex{PDE} is redundant  with the chain rule.  This occurs when the determinant
of the matrix in (\ref{matrix char}) vanishes that is when
\[
\alpha\left( \frac{dy}{ds}\right)^2
-
\beta\left( \frac{dy}{ds}\right)\left( \frac{dx}{ds}\right)
+
\gamma\left( \frac{dx}{ds}\right)^2=0
\]
eliminating the parameter $s$ reduces this equation to the equivalent condition
\[
\alpha\left( \frac{dy}{dx}\right)^2
-
\beta\left( \frac{dy}{dx}\right)
+
\gamma
=0
\]
Formally solving this quadratic for $\frac{dy}{dx}$, we find
\[\frac{dy}{dx}=\frac{\beta\pm\sqrt{\beta^2-4\alpha\gamma}}{2\alpha}\]
This pair of ODE's determine the characteristic curves.  From this equation we
divide into 3 classes each defined with respect to $\beta^2-4\alpha\gamma$.
\begin{enumerate}
\item
\textbf{HYPERBOLIC}\\
$\beta^2-4\alpha\gamma>0$
This gives two families of real characteristic curves.
\item
\textbf{PARABOLIC}\\
$\beta^2-4\alpha\gamma=0$
This gives exactly one family of real characteristic curves.
\item
\textbf{ELLIPTIC}\\
$\beta^2-4\alpha\gamma<0$
This gives no real characteristic equations.
\end{enumerate}
\begin{example}
\underline{The wave equation}
\[c^2\frac{\partial^2u}{\partial x^2}-\frac{\partial^2u}{\partial t^2}=0\]
now equating this with our formula for the characteristics we have
\[\frac{dt}{dx}=\frac{0\pm\sqrt{0+4c^2}}{2}=\pm c\]
this implies that the characteristics are $x+ct=const$ and $x-ct=const$.
This means that the effects travel along the characteristics.\\
\underline{Laplace equation}
\[\frac{\partial^2 u}{\partial x^2} + \frac{\partial^2u}{\partial y^2} =0\]
from this we have $-4(1)(1)< 0$ which implies it is elliptic.\\
This means that information at one point affects all other points.\\
\underline{\addtoindex{Heat equation}}
\[\frac{\partial^2u}{\partial x^2}-\frac{\partial u}{\partial t}=0\]
from this we have $\beta^2-4\alpha\gamma=0$ this implies that the equation is
parabolic thus we have
\[ \frac{\partial t}{\partial x}=0\]
$\diamond$\\
\end{example} 
We can also state that hyperbolic and parabolic are Boundary value problems
and initial value problems.  While, elliptic problems are boundary value problems.
