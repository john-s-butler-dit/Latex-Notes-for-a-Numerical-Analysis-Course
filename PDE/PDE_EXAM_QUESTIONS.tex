\newpage
\chapter{Problem Sheet}
\begin{enumerate}

\item

	\begin{enumerate}
		\item
		State the 3 classes and conditions of 2nd order Partial Differential Equations defined by the characteristic curves.
		
		\item
		Given the non-dimensional form of the heat equation
		\[\frac{\partial U}{\partial t} = \frac{\partial^2 U}{\partial x^2}.\]
		supply sample boundary conditions to specify this problem.\\
		Write a fully implicit scheme to solve this partial differential equation.
		\item
		Derive the local truncation error for the fully implicit method, for the heat equation.
		\item
		Show that the method is unconditionally stable using von Neumann's method.
		%with boundary conditions
		%\[ u(0,t)=u(\pi,t)=0, \ \ 0<t \]
		%and initial conditions
		%\[ u(x,0)=sin(x), 0\leq x \leq \pi.\]
		%Approximate the solution of the heat equation using the Crank-Nicholson method
		%with k=0.01 and $h=\frac{\pi}{10}$ for one time step.\\
		%The exact solution is $u(x,t)=e^{-t}sin(x)$
		

	\end{enumerate}

\item

	\begin{enumerate}
		\item
		State the 3 classes and conditions of 2nd order Partial Differential Equations defined by the characteristic curves.
		\item
		Given the non-dimensional form of the heat equation
		\[\frac{\partial U}{\partial t} = \frac{\partial^2 U}{\partial x^2},\]
		supply sample boundary conditions to specify this problem.\\
		Write an explicit scheme to solve this partial differential equation.
		\item
		Derive the local truncation error for the explicit method, for the heat equation.\\
						
		\item
		Show that the method is consistent, convergent and stable for $\frac{k}{h^2}<\frac{1}{2}$, where k is the step-size in the $t$ direction and $h$ is the step-size
		in the $x$ direction.
		%with boundary conditions
		%\[ u(0,t)=u(\pi,t)=0, \ \ 0<t \]
		%and initial conditions
		%\[ u(x,0)=sin(x), 0\leq x \leq \pi.\]
		%Approximate the solution of the heat equation using the Crank-Nicholson method
		%with k=0.01 and $h=\frac{\pi}{10}$ for one time step.\\
		%The exact solution is $u(x,t)=e^{-t}sin(x)$
		
					
	\end{enumerate}
	\item
		\begin{enumerate}
		\item
		State the 3 classes and conditions of 2nd order Partial Differential Equations defined by the characteristic curves.
		\item
		Given the non-dimensional form of the heat equation
		\[\frac{\partial U}{\partial t} = \frac{\partial^2 U}{\partial x^2},\]
		supply sample boundary conditions to specify this problem.\\
		Write an the Crank-Nicholson method to solve this partial differential equation.
		\item
		Derive the local truncation error for the Crank-Nicholson method, for the heat equation.\\
						
		\item
		Show that the method is unconditionally stable using von Neumann's method.
		%with boundary conditions
		%\[ u(0,t)=u(\pi,t)=0, \ \ 0<t \]
		%and initial conditions
		%\[ u(x,0)=sin(x), 0\leq x \leq \pi.\]
		%Approximate the solution of the heat equation using the Crank-Nicholson method
		%with k=0.01 and $h=\frac{\pi}{10}$ for one time step.\\
		%The exact solution is $u(x,t)=e^{-t}sin(x)$
		
			
	\end{enumerate}
	\item
	\begin{enumerate}
		\item
		Approximate the Poisson equation 
		\[ -\nabla^2U(x,y)=f(x,y) \ \ \ \ \ \ (x,y) \in \Omega=(0,1)\times (0,1) \]
		with boundary conditions
		\[U(x,y) = g(x,y) \ \ \ \ \ \ \ \  (x,y)\in\delta\Omega-boundary \]
		using the five point method.  Sketch how the finite difference scheme may be 
		rewritten in the form $Ax=b$, where A is a sparse
		$N^2\times N^2$ matrix, $b$ is an $N^2$ component matrix and $x$ is an $N^2$
		component vector of unknowns.
		(Assume your 2d discretised grid contains $N$ components in the $x$ and $y$ direction).
		%with boundary conditions
		%\[ u(0,t)=u(\pi,t)=0, \ \ 0<t \]
		%and initial conditions
		%\item
		%Prove that the five-point difference analog $-\nabla^2_h$ is consistent to order 2 with $-\nabla^2$.
		
		\item Prove (DISCRETE MAXIMUM PRINCIPLE).
		if $\nabla^2_hV_{ij}\geq 0$ for all points $(x_i,y_j) \in \Omega_h$, then
		\[ \max_{(x_i,y_j)\in\Omega_h}V_{ij}\leq  \max_{(x_i,y_j)\in\partial\Omega_h}V_{ij}\]
		If $\nabla^2_hV_{ij}\leq 0$ for all points $(x_i,y_j) \in \Omega_h$, then
		\[ \min_{(x_i,y_j)\in\Omega_h}V_{ij}\geq  \min_{(x_i,y_j)\in\partial\Omega_h}V_{ij}\]

		\item
		Hence prove:\\
		Let $U$ be a solution to the Poisson equation and let $w$ be the grid function
		that satisfies the discrete analog
		\[-\nabla_h^2w_{ij}=f_{ij} \ \ \mbox{ for } (x_i,y_j)\in\Omega_h, \]
		\[w_{ij}=g_{ij} \ \ \mbox{ for } (x_i,y_j)\in\partial\Omega_h. \]
		Then there exists a positive constant $K$ such that
		\[||U-w||_{\Omega}\leq KMh^2 \]
		where
		\[ M=\left\{
		\left|\left|\frac{\partial^4 U}{\partial x^4} \right|\right|_{\infty},
		\left|\left|\frac{\partial^4 U}{\partial x^3\partial y} \right|\right|_{\infty},
		,...,
		\left|\left|\frac{\partial^4 U}{\partial y^4} \right|\right|_{\infty}
		\right\}
		\]
		You may assume:\\
		\textbf{Lemma}\\
		If the grid function $V:\Omega_h\bigcup\partial\Omega_h\rightarrow R$ satisfies the boundary condition $V_{ij}=0$ for $(x_i,y_j)\in \partial\Omega_h$, then
		\[||V||_{\Omega}\leq \frac{1}{8}||\nabla_h^2V||_{\Omega} \]
		\end{enumerate}
	



	
	\item
		\begin{enumerate}
		\item
		For a finite difference scheme approximating a partial differential equation of the form
		\[\frac{\partial U}{\partial t}=-a\frac{\partial U}{\partial x}+f(x,t), \ \ x \in R, \ \ \ t>0 \]
		\[ U(x,0)= U_0(x), \ \ \ x\in R \]
		define what is meant by:
		\begin{enumerate}
			\item convergence,
			\item consistency,
			\item stability.
		\end{enumerate}
		\item
		Describe the forward Euler/centered difference method for the transport equation
		and	derive the local truncation error.
		\item
		Define the Courant Friedrichs Lewy condition and state how it is related to stability.
		\item
		Show that the method is stable under the Courant Friedrichs Lewy condition using Von Neumann analysis, you may assume  $f(x,t)=0.$

	\end{enumerate}	
\item
	Consider the second order differential equation 
	\[\frac{d^2u}{dx^2}+u=x \ \]
	with boundary conditions
	\[u(0)=0 \ \ \ \ \ u(1)=0 \]
	\begin{enumerate}
		\item
		Show that the solution $u(x)$ of this equation satisfies the weak form
		\[ \int_{0}^{1} dx \left(-\frac{du}{dx}\frac{dv}{dx}+uv-xv \right) = 0\]
		for all $v(x)$ which are sufficiently smooth and which satisfy
		\[v(0)=0 \ \ \ \ \ v(1)=0 \]
		\item
		By splitting the interval $x\in [0,1]$ into $N$ equal elements of size $h$, where
		$Nh=1$, one can define nodes $x_i$ and finite element shape functions as 
		follows
		\[x_i=ih \]
		\[\phi_i(x)=\left\{\begin{array}{ll} 
		0 & 0\leq x \leq x_{i-1}\\
		\frac{x-x_{i-1}}{h} & x_{i-1}\leq x \leq x_{i}\\
		\frac{x_{i+1}-x}{h} & x_{i}\leq x \leq x_{i+1}\\
		0 & x_{i+1}\leq x \leq 1\\
		\end{array} \right. \]
		A finite element approximation to the differential equation is obtained by approximating $u(x)$ and $v(x)$ with linear combinations of these finite element shape
		functions, $\phi_i$, where
		\[ u_{n} = \sum_{i=1}^{N-1}\alpha_i \phi_i(x) \]
		\[ v_{n} = \sum_{j=1}^{N-1}\beta_j \phi_j(x) \]
		Show that the equation which results from this approximation has the form
		\[K\alpha= F \]
		where K is an $N-1 \times N-1$ sparse matrix, $F$ is an $N-1$ component vector
		and $\alpha$ is an $N-1$ component vector of unknown co-efficient $\alpha_i$.
		\item
		What structure does the matrix $K$ have?
		Evaluate the first component of the main diagonal of $K$.
\end{enumerate}

\end{enumerate}
