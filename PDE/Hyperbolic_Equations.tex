\chapter{Hyperbolic equations}
First-order scalar equation
\begin{equation}
\label{hyper 1}
\begin{array}{cc}
\frac{\partial U}{\partial t} =-a\frac{\partial U}{\partial x}
& x \in R \ \ t >0\\
U(x,0)= U_0(x) & x \in R\end{array} \end{equation}
where $a$ is a positive real number. Its solution is given by
\[U(x,t) = U_0(x-at) \ \ \ \ t \geq 0 \] 
and represents a traveling wave with velocity $a$. The curves $(x(t),t)$ in 
the plane $(x,t)$ are the characteristic curves. They are the straight lines 
$x(t)=x_0+at$, $t >0$.
The solution of (\ref{hyper 1}) remains constant along them.\\
For the more general problem 
 
\begin{equation}
\label{hyper}
\begin{array}{cc}
\frac{\partial U}{\partial t}+ a\frac{\partial U}{\partial x}+a_0 = f 
& x \in R \ \ t >0\\
U(x,0)= U_0(x) & x \in R\end{array} \end{equation}
where 
$a$, $a_0$ and $f$ are given functions of the variables $(x,t)$, the characteristic
curves are still defined as before. In this case the solutions of (\ref{hyper})
satisfy along the characteristics the following differential equation
\[\frac{du}{dt}=f-a_0u \mbox{ on } (x(t),t) \]
\begin{example}
Burgers equation
\[\frac{\partial u}{\partial t}+ u\frac{\partial u}{\partial x} =0 \]
This is a non-trivial non-linear hyperbolic equation. Taking initial condition
\[ u(x,0) = u_0(x) = \left\{ \begin{array}{cc} 
1 & x_0\leq 0 \\
1-x & 0\leq x_0\leq 1 \\
0 & x_0\geq 1 
 \end{array} \right.\]
the characteristic line issuing from the point $(x_0,0)$ is given by
\[ x(t) = x_0+tu_0(x_0) = \left\{ \begin{array}{cc} 
x_0+t& x_0\leq 0 \\
x_0+t(1-x_0) & 0\leq x_0\leq 1 \\
x_0 & x_0\geq 1 
 \end{array} \right.\]
\end{example}
\section{The Wave Equation}
Consider the second-order hyperbolic equation
\begin{equation}
\label{wave}
\frac{\partial^2 U}{\partial t^2} - \gamma \frac{\partial^2 U}{\partial x^2}=f \ \ \ x \in (\alpha,\beta), \ \ t>0\end{equation}
with initial data
\[U(x,0) = u_0(x) \mbox{   and   } \frac{\partial U}{\partial t}(x,0)=v_0(x), \ \ x\in (\alpha,\beta) \]
and boundary data
\[U(\alpha,t) = 0 \mbox{   and   } U(\beta,t)=0, \ \ t>0 \]
In this case, $U$ may represent the transverse displacement of an elastic vibrating 
string of length $\beta-\alpha$, fixed at the endpoints and $\gamma$ is a coefficient
depending on the specific mass of the string and its tension. The spring is subject to a vertical force of density $f$.\\

The functions $u_0(x)$ and $v_0(x)$ denote respectively the initial displacement and initial velocity of the string.\\
The change of variables
\[\omega_1=\frac{\partial U}{\partial x}, \ \ \ \omega_2 =\frac{\partial U}{\partial t} \]
transforms (\ref{wave}) into 
\[
\frac{\partial \hat{\omega}}{\partial t} +A\frac{\partial \hat{\omega}}{\partial x}= \mathbf{0}
\]
where
\[\hat{\omega}=\left[\begin{array}{c}\omega_1\\ \omega_2\end{array} \right] \]
Since the initial conditions are $\omega_1(x,0)=u'_0(x)$ and $\omega_2(x,0)=v_0(x)$.\\
\subsection*{Aside}
Notice that replacing $\frac{\partial^2 u}{\partial t^2}$ by $t^2$, $\frac{\partial^2u}{\partial x^2}$ by $x^2$ and $f$ by 1, the wave equation becomes
\[t^2-\gamma^2 x^2=1 \]
which represents an hyperbola in $(x,t)$ plane. Proceeding analogously in the case
of the heat equation we end up with 
\[t-x^2=1 \]
which represents a parabola in the $(x,t)$ plane. Finally, for the Poisson
equation we get
\[x^2+y^2=1 \]
which represents an ellipse in the $(x,y)$ plane.\\
Due to the geometric interpretation above, the corresponding differential
operators are classified as hyperbolic, parabolic and elliptic.
\section{Finite Difference Method for Hyperbolic equations}
As always we discretise the domain by space-time finite difference.  To this aim,
the half-plane $\{(x,t): -\infty <x<\infty, t>0 \}$ is discretised by choosing
a spatial grid size $\Delta x$, a temporal step $\Delta t$ and the grid
points $(x_j,t^n)$ as follows
\[x_j=j\Delta x  \ \ \ \ j \in Z, \ \ t^n=n\Delta t \ \ \ \ n \in N \]
and let 
\[\lambda =\frac{\Delta t}{\Delta x}. \]
\subsection{Discretisation of the scalar equation }
Here are some explicit methods 
\begin{itemize}
\item
Forward Euler/centered
\[ u^{n+1}_j=u^{n}_{j}-\frac{\lambda}{2}a(u^{n}_{j+1}-u^{n}_{j-1})\]
\item
Lax-Friedrichs
\[ u^{n+1}_j=\frac{u^{n}_{j+1}+u_{j-1}^n}{2}-\frac{\lambda}{2}a(u^{n}_{j+1}-u^{n}_{j-1})\]
\item
Lax-Wendroff
\[ u^{n+1}_j=u^{n}_{j}-\frac{\lambda}{2}a(u^{n}_{j+1}-u^{n}_{j-1})+\frac{\lambda}{2}a^2(u^{n}_{j+1}-2u_{j}^n+u^{n}_{j-1})\]
\item
Upwind
\[ u^{n+1}_j=u^{n}_{j}-\frac{\lambda}{2}(u^{n}_{j+1}-u^{n}_{j-1})+\frac{\lambda}{2}|a|(u^{n}_{j+1}-2u_{j}^n+u^{n}_{j-1})
\]
\end{itemize}

The last three methods can be obtained from the forward Euler/centered method by
adding a term proportional to a numerical approximation of a second derivative term so that they can be written in the equivalent form
\[ u^{n+1}_j=u^{n}_{j}-\frac{\lambda}{2}a(u^{n}_{j+1}-u^{n}_{j-1})+\frac{1}{2}k\frac{(u^{n}_{j+1}-2u_{j}^n+u^{n}_{j-1})}{(\Delta x)^2}\]
where $k$ is an artificial viscosity term.\\
An example of an implicit method is the backward Euler/ centered scheme
\[u^{n+1}_j+\frac{\lambda}{2}a(u_{j+1}^{n+1} - u_{j-1}^{j+1})=u_{j}^n \]

\section{Analysis of the Finite Difference Methods}
\section{Consistency}
A numerical method is convergent if 
\[\lim_{\Delta t,\Delta x \rightarrow 0} \max_{j,n}|U(x_j,t^n)-w_j^n| \] 
The local truncation error at $x_j,t^n$ is defined as
\[\tau_j^n = L(U^n_j) \]
the truncation error is 
\[\tau(\Delta t, \Delta x) = \max_{j,n}|\tau^n_j| \]
When $\tau(\Delta t, \Delta x)$ goes to zero as $\Delta t$ and $\Delta x$ tend to
zero independently is said to be consistent.
\section{Stability}
\section{Courant Freidrich Lewy Condition}
A method is said to be stable if, for any $T$ here exist a constant $C_T>0$
and $\delta_0$ such that
\[|| \mathbf{u}^n ||_{\Delta}\leq C_T ||\mathbf{u}^0||_{\Delta} \]
for any $n$ such that $n\Delta t \leq T$ and for any $\Delta t, \Delta x$ such that $0 < \Delta t \leq \delta_0,0 < \Delta x \leq \delta_0$. We have denoted by
$ 
|| . ||_{\Delta}$ a suitable discrete norm.

Forward Euler/centered
\[ u^{n+1}_j=u^{n}_{j}-\frac{\lambda}{2}a(u^{n}_{j+1}-u^{n}_{j-1})\]
Truncation error
\[O(\Delta t , (\Delta x)^2) \]
For an explicit method to be stable we need
\[|a\lambda| = \left|a\frac{\delta t}{\delta x}\right| \leq 1 \]
this is known as the Courant Freidrich Lewy condition.\\
Using  Von Neumann stability analysis we can show that the method is stable
under the Courant Freidrich Lewy condition.\\

\subsection{von Neumann stability for the Forward Euler}
\[u_j^n = e^{i\beta j \Delta x} \xi^n \]
where 
\[\xi = e^{\alpha \Delta t}  \]
It is sufficient to show 
\[|\xi| \leq 1 \]
\[ \xi^{n+1}e^{i\beta(j)\Delta x}=\xi^{n}e^{i\beta(j)\Delta x}+\frac{\lambda}{2}a(\xi^{n}e^{i\beta(j+1)\Delta x}-\xi^{n}e^{i\beta(j-1)\Delta x})\]
\[ \xi = 1-\frac{\lambda}{2}a(e^{i\beta\Delta x}-e^{-i\beta\Delta x})\]
\[ \xi = 1-i\frac{\lambda}{2}a(2sin(\beta \Delta x)) \]
\[ \xi = 1-i\lambda a(sin(\beta \Delta x)) \]
\[ |\xi| = \sqrt{1+(\lambda a(sin(\beta \Delta x)))^2} \]
Hence
\[ \xi>1\]
therefore the method is unstable for the Courant Freidrich Lewy.

\subsection{von Neumann stability for the Lax-Friedrich}


\[ \xi^{n+1}e^{i\beta(j)\Delta x}=\frac{\xi^{n}e^{i\beta(j+1)\Delta x}+\xi^{n}e^{i\beta(j-1)\Delta x}}{2}+\frac{\lambda}{2}a(\xi^{n}e^{i\beta(j+1)\Delta x}-\xi^{n}e^{i\beta(j-1)\Delta x})\]
\[ \xi = \frac{e^{i\beta\Delta x}-e^{-i\beta\Delta x}}{2}+\frac{\lambda}{2}a(e^{i\beta\Delta x}-e^{-i\beta\Delta x})\]
\[ \xi = \frac{1+\lambda a}{2}e^{i\beta\Delta x}+\frac{1-\lambda a}{2}e^{-i\beta\Delta x} \]
\[ \xi = \cos(\beta \Delta x)+i\lambda a\sin(\beta \Delta x) \]
\[ |\xi|^2 \leq (\cos(\beta \Delta x))^2+(a \lambda)^2(\sin(\beta \Delta x))^2 \]
Hence
\[ \xi<1\]
for $a \lambda \leq 1$.